\documentclass{ccg-topic}

\topic{Formal Group Laws I}

\institution{University of Colorado}
\coursenum{Topology Seminar}
\coursename{Chromatic Homotopy Theory}
\semester{Fall 2019}
\author{ccg}
\date{\today}
\email{colton.grainger@colorado.edu}
\thanks{}

\newcommand{\Th}[1]{\mathrm{Th}\paren{#1}} 
\begin{document}
\frontstuff

\newcommand{\FGL}{\mathrm{FGL}} 

\begin{note}[Wikipedia Biography]
    \label{rem:wikipedia_biography}
    Michel Paul Lazard (5 December 1924 – 15 September 1987) was a French mathematician who worked in the theory of Lie groups in the context of p-adic analysis. His work ``took on a life of its own'' in the hands of Daniel Quillen in the late 20th century. Quillen's discovery, that ``a ring Lazard used to classify formal group laws '' was isomorphic to an ``important ring in topology'', lead to the subject of chromatic homotopy theory.

Lazard's self-contained treatise on one-dimensional formal groups (of which formal group laws are a ``coordinate dependent'' manifestation) also birthed the field of p-divisible groups.

His major contributions include: 
\begin{enumerate}
\item The classification of $p$-adic Lie groups: every $p$-adic Lie group is a closed subgroup of $\GL_n(Z_p)$
\item The classification of (1-dimensional commutative) formal groups.
\item That the universal formal group law coefficient ring (Lazard's universal ring) is a polynomial ring.
\item The concept of "analyseurs" (which have been reinvented by J. Peter May under the name operads).
\end{enumerate}
\end{note}

A similar statement is true of ordinary cohomology and the formal group law
one gets is the additive one; this is a restatement of the fact that the first Chern
class of a tensor product of complex line bundles is the sum of the first Chern
classes of the factors. One can play the same game with complex K-theory and get
a multiplicative formal group law.
CP ∞ is a good test space for both complex cobordism and K-theory. One
can analyze the algebra of operations in both theories by studying their behavior
in CP ∞ (see Adams [5]) in the same way that Milnor [2] analyzed the mod (2)
Steenrod algebra by studying its action on H ∗ (RP ∞ ; Z/(2)). (See also Steenrod
and Epstein [1].)

The formal group law of 1.3.3 is not as simple as the ones for ordinary co-
homology or K-theory; it is complicated enough to have the following universal
property.
1.3.4. Theorem (Quillen [2]). For any formal group law F over any commuta-
tive ring with unit R there is a unique ring homomorphism θ : M U ∗ (pt) → R such
that F (x, y) = θF U (x, y).

We remark that the existence
of such a universal formal group law is a triviality.
Simply write F (x, y) =
a i,j x i y i and let L = Z[a i,j ]/I, where I is the ideal
generated by the relations among the a i,j imposed by the definition 1.3.1 of an
formal group law. Then there is an obvious formal group law over L having the
universal property. Determining the explicit structure of L is much harder and was
first done by Lazard [1]. Quillen’s proof of 1.3.4 consisted of showing that Lazard’s
universal formal group law is isomorphic to the one given by 1.3.3.
Once Quillen’s Theorem 1.3.4 is proved, the manifolds used to define complex
bordism theory become irrelevant, however pleasant they may be. All of the ap-
plications we will consider follow from purely algebraic properties of formal group
laws. This leads one to suspect that the spectrum M U can be constructed some-
how using formal group law theory and without using complex manifolds or vector
bundles. Perhaps the corresponding infinite loop space is the classifying space for
some category defined in terms of formal group laws. Infinite loop space theorists,
where are you?

\begin{defn}
    \label{defn:formal_group_law}
    A \term{formal group law} (FGL) over a ring $R \in \cat{CRing}$ is a \term{formal power series} \[F(x,y) = \sum_{\substack{n\ge0\\i+j=n}}c^i_j x^iy_j \in R[[x,y]]\] that \term{formally satisfies} the axioms for a commutative group operation with $0$ as the identity element. 

Equivalently, $F \in R[[x,y]]$ is a formal group law if and only if
\begin{description}
    \item [F is local] $F(x,0) = x \in R[[x]]$
    \item [F is symmetric] $F(x,y) = F(y,x) \in R[[x,y]]$
    \item [F has unique iterated brackets] $F(F(x,y),z) = F(x, F(y,z)) \in R[[x,y,z]]$
    \item [$x$ has an inverse] There is a power series $m(x) \in R[[x]]$ such that $m(0)=0$ and $F(x,m(x)) = 0$.
\end{description}
\end{defn}

In the last two conditions, we need to substitute power series into another. This leads to nonsense if the power series involved have nonzero constant terms, since we have no notion of convergence in the ring $R$. (Consider substituting $1$ for $x$ and $y$, then working with the coefficient $F(1,1) = \sum c^i_j$ in the $0$th degree. Absurd.) However, if the constant terms are zero, then there is no problem in expanding everything out formally. \cite{Str11}

\begin{ex}[Concrete Examples from Neil Strickland]
    \label{ex:concrete}
\hfill
\begin{enumerate}
    \item The simplest example is \[F(x, y) = x + y.\] This is called the additive $\FGL$. It can be defined over any ring $R$.
\item If $u \in R$ then we can take \[F(x,y) = x + y + uxy\] so that
\begin{equation*}
    1+u(x+_F y) = (1+ux)(1+uy).
\end{equation*}
In the case $u = 1$, this is called the multiplicative $\FGL$. It can again be defined over any ring $R$. This $\FGL$ should be reminiscent of ``changing coordinates'' to obtain an additive group law from a one that was multiplicative, e.g., say the multiplicative product in the ring $R$ is \[G(a,b) = ab\]. Then change coordinates so that $a = 1+x$, $b=1+y$, and obtain the $\FGL$ \[F = 1+G \qq{such that} F(x,y) = x + y + xy.\]
      
\item  If $c$ is an invertible element of $R$ then we can define 
\begin{equation*}
    F(x, y) = \frac{x+y}{1+\frac{xy}{c^2}}
\end{equation*}
We call this the Lorentz $\FGL$; it is the formula for relativistic addition of parallel velocities, where $c$ is the speed of light. We are implicitly using the fact that $(1 + \frac{xy}{c^2})$ is invertible in $R[[x, y]]$, with inverse \[\sum_{k\ge0 } \paren{\frac{-xy}{c^2}}^2.\]
\end{enumerate}
\end{ex}

We now prove some basic lemmas, as practice in the use of formal power series.

\begin{lem}[Terms of small order]
    \label{lem:characterizing_terms_of_small_order}
A $\FGL$ is of the form \[x+_F y = x + y + \text{higher order terms}.\] That is, if $F$ is an $\FGL$, then $F(x,y) = x + y \pmod{xy}$.
\end{lem}
\begin{proof}
Let $F(x,y) = \sum_{i,j \ge 0} c^i_j x^iy_j$ for some coefficients $c^i_j \in \R$ (image an infinite array). So that $x +_F 0 = x$, we must have $c^i_0 = 0$ except for $c^1_0 = 1$. So that $x+_F y = y +_F x$, we must have $c^0_j = 0$ except for $c^0_1 = 1$. Hence
\[
    F(x,y) = x + y + xy\sum_{i,j > 0}c^i_j x^{i-1}y_{j-1}.
\]
\end{proof}

Imagining an infinite array of coefficients, we have used that ``F is local'' to determine the only interesting coefficients are past the $0$th column and row. Taking this ``interesting'' subarray helps justify the indexing conventions when we impose the following grading:
\[
    \deg c^i_j = i + j -1
\]
That ``F is symmetric in its arguments'' just forces the coefficients in the array to satisfy $c^i_j = c^j_i$. That ``F has unique iterated brackets'' alludes to the machinations necessary to prove (Michel) Lazard's theorem. historical application

\begin{ex}[deRham cohomology algebra]
    \label{ex:derham_cohomology_algebra}
Recall that a graded algebra over $\R$ is a pair $((A^k)_{k \in \Z}, m)$ where
$(A^k)$ is a collection of $R$-vector spaces $A_k$, and $m$ is a linear map
$m \colon (\oplus_k A^k) \otimes (\oplus_k A^k) \to \oplus_k A^k$ such that $m$ maps $A^k \otimes A^l$ to $A^{k+l}$.
    Another example is the polynomial ring in one generator, $\R[x]$, where $A^k$ is the vector space of homogeneous degree $k$ polynomials. A variant is $\R[y]$, where $A^{2k}$ is the vector space of homogeneous degree $k$ polynomials, and $A^\text{odd}$ is zero. (So $y$ is in degree $2$.) Depending on your taste, $\R[y]$ is isomorphic to the deRham cohomology ring of $\CP^\infty$. If your taste is different, then the \term{power series ring} $\R[[y]]$ is rather isomorphic to the deRham cohomology ring of $\CP^\infty$.
\end{ex}

\section{Complex Co-bordism}

Suppose $E$ is a generalized cohomology theory, and $X$ is a paracompact space.

\begin{defn}[$E$-orientation]
    \label{defn:_e_orientation}
    A vector bundle $V \xrightarrow{p} X$ over a paracompact space $X$ is $E$-orientable if there exists $x$ in the $n$-th degree of the cohomology ring $E^n \paren{\Th X}$ of the \term{Thom Space} $\Th X$ of $X$ \TODO
\end{defn}

\section{Formal Group Laws}

(6) In algebraic topology, one can consider a number of complex-orientable generalised cohomology
theories. Such a theory assigns to each space X a graded ring E∗X, subject to various axioms. If
L is a complex line bundle over X, one can define an Euler class e(L) ∈ E∗X, which is a useful
invariant of L. There is a formal group law F over E∗(point) such that e(L ⊗ M) = F(e(L), e(M)).
In the case of ordinary cohomology, we get the additive FGL. In the case of complex K-theory,
we get the multiplicative FGL. In the case of complex cobordism, we get Lazard’s universal FGL
(Quillen’s theorem). This is the start of a very deep relationship between formal groups and the
algebraic aspects of stable homotopy theory.

\bibliography{/home/colton/coltongrainger.bib}
\bibliographystyle{alpha}
\end{document}
